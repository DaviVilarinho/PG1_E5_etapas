\documentclass[
	% -- opções da classe memoir --
	12pt,				% tamanho da fonte
	openright,			% capítulos começam em pág ímpar (insere página vazia caso preciso)
	oneside,			% para impressão em verso e anverso. Oposto a oneside
	a4paper,			% tamanho do papel. 
	% -- opções da classe abntex2 --
	chapter=TITLE,		% títulos de capítulos convertidos em letras maiúsculas
	%section=TITLE,		% títulos de seções convertidos em letras maiúsculas
	subsection=TITLE,	% títulos de subseções convertidos em letras maiúsculas
	%subsubsection=TITLE,% títulos de subsubseções convertidos em letras maiúsculas
	% -- opções do pacote babel --
	english,			% idioma adicional para hifenização
	brazil,				% o último idioma é o principal do documento
	]{abntex2}

% ---
% PACOTES
% ---

% ---
% Pacotes fundamentais 
% ---
\usepackage{lmodern}			% Usa a fonte Latin Modern
\usepackage[T1]{fontenc}		% Selecao de codigos de fonte.
\usepackage[utf8]{inputenc}		% Codificacao do documento (conversão automática dos acentos)
\usepackage{indentfirst}		% Indenta o primeiro parágrafo de cada seção.
\usepackage{color}				% Controle das cores
\usepackage{graphicx}			% Inclusão de gráficos
\usepackage{microtype} 			% para melhorias de justificação
% ---

%--------------Pacotes de citações
\usepackage[brazilian,hyperpageref]{backref}	 % Paginas com as citações na bibl
\usepackage[alf]{abntex2cite}	% Citações padrão ABNT

\usepackage{hyperref}

% --- 
% CONFIGURAÇÕES DE PACOTES
% --- 

% ---
% Configurações do pacote backref
% Usado sem a opção hyperpageref de backref
\renewcommand{\backrefpagesname}{Citado na(s) página(s):~}
% Texto padrão antes do número das páginas
\renewcommand{\backref}{}
% Define os textos da citação
\renewcommand*{\backrefalt}[4]{
	\ifcase #1 %
		Nenhuma citação no texto.%
	\or
		Citado na página #2.%
	\else
		Citado #1 vezes nas páginas #2.%
	\fi}%
% ---

% alterando o aspecto da cor azul
\definecolor{blue}{RGB}{41,5,195}

% informações do PDF
\makeatletter
\hypersetup{
     	%pagebackref=true,
		pdftitle={\@title}, 
		pdfauthor={\@author},
    	pdfsubject={\imprimirpreambulo},
	    pdfcreator={LaTeX with abnTeX2},
		pdfkeywords={abnt}{latex}{abntex}{abntex2}{plano de trabalho}, 
		colorlinks=true,       		% false: boxed links; true: colored links
    	linkcolor=blue,          	% color of internal links
    	citecolor=blue,        		% color of links to bibliography
    	filecolor=magenta,      		% color of file links
		urlcolor=blue,
		bookmarksdepth=4
}
\makeatother
% --- 

% --- 
% Espaçamentos entre linhas e parágrafos 
% --- 

% O tamanho do parágrafo é dado por:
\setlength{\parindent}{1.3cm}

% Controle do espaçamento entre um parágrafo e outro:
\setlength{\parskip}{0.2cm}  % tente também \onelineskip

% ---
% compila o indice
% ---
\makeindex
% ---

%-----------------------------------------
\begin{document}

% Retira espaço extra obsoleto entre as frases.
\frenchspacing 

\textbf{Graduação em Sistemas de Informação - UFU}
%\textbf{Graduação em Ciência da Computação - UFU}

\textbf{Disciplina:} FACOM31701 - Trabalho de Conclusão de Curso 1 
%\textbf{Disciplina:} GBC072 - Projeto de Graduação 1 

\textbf{Orientador(a):} Nome do orientador(a)

\textbf{Nome:} Fulano de Tal

% ------------------ Título -----------------------
\section*{\centerline{\Large \textbf{Entrega 5 (E5) - Método de pesquisa e manipulação do modelo}}}
% -------------------------------------------------

%------------------- Corpo de dados --------------
\vspace*{0.5cm}

\textbf{1) Identifique, juntamente com o seu orientador(a), quais são as principais etapas do método de pesquisa do seu projeto. De acordo com o livro do Raul Wazlawick, ``O método consiste na sequência de passos necessários para demonstrar que o objetivo proposto foi atingido.'' e ``O método deve então indicar se protótipos serão desenvolvidos, se modelos teóricos serão construídos, quais experimentos eventualmente serão realizados, como os dados serão organizados e comparados, e assim por diante, dependendo do tipo de trabalho.'' O objetivo aqui é criar uma lista que sumarize a sequência de passos que será adotada.}

\textbf{Resposta:}

\textbf{2) Use o modelo disponibilizado pelas coordenações para criar quatro capítulos: 1) Introdução, 2) Fundamentação teórica, 3) Trabalhos relacionados e 4) Método. Os textos das entregas E2, E3, E4 e E5 formarão os capítulos 1, 2, 3 e 4, respectivamente. Aproveite este momento para fazer correções sugeridas pelos orientadores e também pelo professor da disciplina. A resposta da questão 2) deve ser o PDF do modelo. }

Links para os modelos:
\begin{enumerate}
    \item \url{https://facom.ufu.br/system/files/conteudo/template_tcc_bsi_0.zip}
    \item \url{https://facom.ufu.br/system/files/conteudo/template_tcc_bcc.zip}
\end{enumerate}

\bibliography{ref}

\end{document}
